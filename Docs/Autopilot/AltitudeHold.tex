%
% `AltitudeHold.tex' -- describes the FGFS Altitude Hold
%
%  Written by Curtis Olson.  Started December, 1997.
%
% $Id$
%------------------------------------------------------------------------


\documentclass[12pt]{article}

\usepackage{anysize}
\papersize{11in}{8.5in}
\marginsize{1in}{1in}{1in}{1in}

\usepackage{amsmath}

\usepackage{epsfig}

\usepackage{setspace}
\onehalfspacing

\usepackage{url}


\begin{document}


\title{
  Flight Gear Autopilot: \\
  Altitude Hold Module
}


\author{
  Curtis Olson \\ 
  (\texttt{curt@me.umn.edu})
}


\maketitle

\section{Introduction}

Working on scenery creation was becoming stressful and overwhelming.
I needed to set it aside for a few days to let my mind regroup so I
could mount a fresh attack.

As a side diversion I decided to take a stab at writing an altitude
hold module for the autopilot system and in the process hopefully
learn a bit about control theory.


\section{Control Theory 101}

Before I get too far into this section I should state clearly and up
front that I am not a ``controls'' expert and have no formal training
in this field.  What I say here is said \textit{to the best of my
knowledge.}  If anything here is mistaken or misleading, I'd
appreciate being corrected.  I'd like to credit my boss, Bob Hain, and
my coworker, Rich Kaszeta, for explaining this basic theory to me, and
answering my many questions.

The altitude hold module I developed is an example of a PID
controller.  PID stands for proportional, integral, and derivative.
These are three components to the control module that will take our
input variable (error), and calculate the value of the output variable
required to drive our error to zero.

A PID needs an input variable, and an output variable.  The input
variable will be the error, or the difference between where we are,
and where we want to be.  The output variable is the position of our
control surface.

The proportional component of the PID drives the output variable in
direct proportion to the input variable.  If your system is such that
the output variable is zero when the error is zero and things are
mostly linear, you usually can get by with proportional control only.
However, if you do not know in advance what the output variable will
be when error is zero, you will need to add in a measure of integral
control.

The integral component drives the output based on the area under the
curve if we graph our actual position vs. target position over time.

The derivative component is something I haven't dealt with, but is
used to drive you towards your target value more quickly.  I'm told
this must be used with caution since it can easily yield an unstable
system if not tuned correctly.

Typically you will take the output of each component of your PID and
combine them in some proportion to produce your final output.

The proportional component quickly stabilizes the system when used by
itself, but the system will typically stabilize to an incorrect value.
The integral component drives you towards the correct value over time,
but you typically oscillate over and under your target and does not
stabilize quickly.  However, each of these provides something we want.
When we combine them, they offset each others negatives while
maintaining their desirable qualities yielding a system that does
exactly what we want.

It's actually pretty interesting and amazing when you think about it.
the proportional control gives us stability, but it introduces an
error into the system so we stabilize to the wrong value.  The
integral components will continue to increase as the sum of the errors
over time increases.  This pushes us back the direction we want to
go.  When the system stabilizes out, we find that the integral
component precisely offsets the error introduced by the proportional
control.

The concepts are simple and the code to implement this is simple.  I
am still amazed at how such a simple arrangement can so effectively
control a system.


\section{Controlling Rate of Climb}

Before we try to maintain a specific altitude, we need to be able to
control our rate of climb.  Our PID controller does this through the
use of proportional and integral components.  We do not know in
advance what elevator position will establish the desired rate of
climb.  In fact the precise elevator position could vary as external
forces in our system change such as atmospheric density, throttle
settings, aircraft weight, etc.  Because an elevator position of zero
will most likely not yield a zero rate of climb, we will need to add
in a measure of integral control to offset the error introduced by the
proportional control.

The input to our PID controller will be the difference (or error)
between our current rate of climb and our target rate of climb.  The
output will be the position of the elevator needed to drive us towards
the target rate of climb.

The proportional component simply sets the elevator position in direct
proportion to our error.
\[ \mathbf{prop} = K \cdot \mathbf{error} \]

The integral component sets the elevator position based on the sum of
these errors over time.  For a time, $t$
\[ \mathbf{integral} = K \cdot \int_{0}^{t} { \mathbf{error} \cdot \delta t } \]

I do nothing with the derivative component so it is always zero and
can be ignored.

The output variable is just a combination of the proportional and
integral components.  $w_{\mathit{prop}}$ and $w_{\mathit{int}}$ are
weighting values.  This allows you to control the contribution of each
component to your final output variable.  In this case I found that
$w_{\mathit{prop}} = 0.9$ and $w_{\mathit{int}} = 0.1$ seemed to work
quite well.  Too much integral control and your system won't
stabilize.  Too little integral control and your system takes
excessively long to stabilize.
\[
\mathbf{output} = w_{\mathit{prop}} \cdot \mathbf{prop} +
                     w_{\mathit{int}} \cdot \mathbf{int}
\]

We are trying to control rate of climb with elevator position, so the
output of the above formula is our elevator position.  Using this
formula to set a new elevator position each iteration quickly drives
our climb rate to the desired value.


\section{Controlling Altitude}

Now that we have our rate of climb under control, it is a simple
matter to leverage this ability to control our absolute altitude.

The input to our altitude PID controller is the difference (error)
between our current altitude and our goal altitude.  The output is the
rate of climb needed to drive our altitude error to zero.

Clearly, our climb rate will be zero when we stabilize on the target
altitude.  Because our output variable will be zero when our error is
zero, we can get by with only a proportional control component.

All we need to do is calculate a desired rate of climb that is
proportional to how far away we are from the target altitude.  This is
a simple proportional altitude controller that sits on top of our
slightly more complicated rate of climb controller.

\[
\mathbf{target\_climb\_rate} = K \cdot ( \mathbf{target\_altitude} - 
	\mathbf{current\_altitude} )
\]

Thus we use the difference in altitude to determine a climb rate and
we use the desired climb rate to determine elevator position.


\section{Parameter Tuning}

I've explained the basics, but there is one more thing that is
important to mention.  None of the above theory and math is going to
do you a bit of good for controlling your system if all your
parameters are out of whack.  In fact, parameter tuning is often the
trickiest part of the whole process.  Expect to spend a good chunk of
your time tweaking function parameters to fine tune the behavior and
effectiveness of your controller.


\end{document}


%------------------------------------------------------------------------
% $Log$
% Revision 1.1  1999/04/05 21:32:34  curt
% Initial revision
%
% Revision 1.1  1999/03/09 19:09:41  curt
% Initial revision.
%
