\documentclass[12pt,titlepage]{article}

\usepackage{anysize}
\papersize{11in}{8.5in}
\marginsize{1in}{1in}{1in}{1in}


\begin{document}

Here is my attempt to organize descriptions of the various LaRCsim
files required to implement the equations of flight.  99\% of the
following text is copied straight out of email from Bruce, source code
comments, or the LaRCsim manual.

\section{Core LaRCsim Header Files}

\begin{description}
  \item[ls\_generic.h:]1 LaRCSim generic parameters header file.  Defines
    the ``GENERIC'' structure which holds the current value of the
    flight model parameters and states.

  \item[ls\_types.h:] LaRCSim type definitions header file.  Defines
    the following types: SCALAR, VECTOR\_3, and DATA.

  \item[ls\_constants.h:] LaRCSim constants definition header file.
    Defines various constants and various units conversions.
    
  \item[ls\_sim\_control.h:] LaRCSim simulation control parameters
    header file
\end{description}


\section{Core LaRCsim Routines}

The following share the ls\_generic.h, ls\_types.h, and ls\_constants.h
header files.

\begin{description}
  \item[ls\_accel.c:] ls\_accel() sums the forces and moments from aero,
    engine, gear, transfer them to the center of gravity, and calculate
    resulting accelerations.
    
  \item[ls\_step.c:] ls\_step() Integration routine for equations of
    motion (vehicle states.)  Integrates accels $\rightarrow$
    velocities and velocities $\rightarrow$ positions.
    
  \item[ls\_aux.c:] ls\_aux() Takes the new state information
    (velocities and positions) and calculates other information, like
    Mach, pressures \& temps, alpha, beta, etc. for the new state. It
    does this by calling atmos\_62() ls\_geodesy() and ls\_gravity().

  \item[atmos\_62.c] atmos\_62() 1962 standard atmosphere table lookups.
    
  \item[ls\_geodesy.c] ls\_geoc\_to\_geod(lat\_geoc, radius, lat\_geod, alt,
    sea\_level\_r) ls\_geod\_to\_geoc(lat\_geod, alt, sl\_radius, lat\_geoc)
    since vehicle position is in geocentric lat/lon/radius, this
    routine calculates geodetic positions lat/lon/alt ls\_gravity -
    calculates local gravity, based on latitude \& altitude.
    
  \item[ls\_gravity:] ls\_gravity( SCALAR radius, SCALAR lat, SCALAR
    *gravity ) Gravity model for LaRCsim.
\end{description}


\section{Secondary LaRCsim Routines}

The following routines help manage the simulation

\begin{description}
  \item[ls\_model.c:] ls\_model() Model loop executive.  Calls the user
    supplied routines: inertias(), subsystems(), engine(), aero(), and
    gear().
    
  \item[default_model_routines.c:] Provides stub routines for the
    routines that are normally provided by the user. 
\end{description}


\section{Navion Specific Routines}

\begin{description}
  \item[ls\_cockpit.h:] Header for cockpit IO.  Stores the current
    state of all the control inputs.
    
  \item[navion\_aero.c:] aero() Linear aerodynamics model.  Initializes
    all the specific parameters if not initialized.  The expected
    outputs from aero() are the aerodynamic forces and moments about
    the reference point, in lbs and ft-lbs, respectively, being stored
    in the F\_aero\_v and M\_aero\_v vectors.
    
  \item[navion\_engine.c:] engine() Calculate the forces generated by
    the engine.
    
  \item[navion\_gear.c:] gear() Landing gear model for example simulation.
    
  \item[navion\_init.c:] model\_init() Initializes navion math model
\end{description}

\end{document}