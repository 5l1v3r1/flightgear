%%
%% getstart.tex -- Flight Gear documentation: Installation and Getting Started
%% Chapter file
%%
%% Written by Michael Basler, started September 1998.
%%
%% Copyright (C) 1999 Michael Basler (pmb@knUUt.de)
%%
%%
%% This program is free software; you can redistribute it and/or
%% modify it under the terms of the GNU General Public License as
%% published by the Free Software Foundation; either version 2 of the
%% License, or (at your option) any later version.
%%
%% This program is distributed in the hope that it will be useful, but
%% WITHOUT ANY WARRANTY; without even the implied warranty of
%% MERCHANTABILITY or FITNESS FOR A PARTICULAR PURPOSE.  See the GNU
%% General Public License for more details.
%%
%% You should have received a copy of the GNU General Public License
%% along with this program; if not, write to the Free Software
%% Foundation, Inc., 675 Mass Ave, Cambridge, MA 02139, USA.
%%
%% $Id: getstart.tex,v 0.20 1999/06/04 michael
%% (Log is kept at end of this file)

%%%%%%%%%%%%%%%%%%%%%%%%%%%%%%%%%%%%%%%%%%%%%%%%%%%%%%%%%%%%%%%%%%%%%%%%%%%%%%%%%%%%%%%%%%%%%%%%%
\chapter{Landing: Some further thoughts before leaving the plane\label{landing}}
%%%%%%%%%%%%%%%%%%%%%%%%%%%%%%%%%%%%%%%%%%%%%%%%%%%%%%%%%%%%%%%%%%%%%%%%%%%%%%%%%%%%%%%%%%%%%%%%%
\markboth{\thechapter.\hspace*{1mm}
LANDING}{\thesection\hspace*{1mm} THOSE, WHO DID THE WORK}


\section{Those, who did the work}

Did you enjoy the flight? In case you did, don't forget those who devoted hundreds of
hours to that project. All of this work is done on a voluntary basis within spare time,
thus bare with the \Index{programmers} in case something does not work the way you want
it to. Instead, sit down and write them a kind (!) letter proposing what to change.
Alternatively, you can subscribe to the \FlightGear \Index{mailing lists} and contribute
your thoughts there. Instructions to do so can be found under

 \web{http://www.flightgear.org/mail.html}.

Essentially there are two lists, one of which being mainly for the developers and the
other one for end users.
\medskip

 \noindent
These are the people who did the job (This information was
essentially taken from the file \texttt{Thanks} accompanying the
code):
 \medskip

\noindent Raul Alonzo\index{Alonzo, Raul} (\mail{amil@las.es})\\ Author of Ssystem and
moon texture.
 \medskip


\noindent Michele America\index{America, Michele}
(\mail{nomimarketing@mail.telepac.pt})\\
  Contributed to the \Index{HUD} code.
 \medskip

\noindent Steve Baker\index{Baker, Steve} (\mail{sjbaker@hti.com})\\
  Author of \Index{PLIB}, a graphics/audio/joystick interface written entirely on top of
    \Index{OpenGL}/\-\Index{GLUT} used in \FlightGear. An immense amount of coaching and tutelage,
    both on the subjects of flight simulation and \Index{OpenGL}.  It has been
    his comments and thoughts that have prompted the implementation of
    most of the more sophisticated features of \FlightGear{\hspace{-2mm}}.
 \medskip

\noindent Michael Basler\index{Basler, Michael} (\mail{pmb@knUUt.de})\\
 Coauthor of Installation and Getting Started (together with Bernhard
 Buckel).
\medskip

\noindent John S. Berndt\index{Berndt, John, S.} (\mail{jsb@hal-pc.org})\\
 Working on a complete C++rewrite/reimplimentation of the core FDM.
  Initially he is using X15 data to test his code, but once things are
  all in place we should be able to simulator arbitrary aircraft.
\medskip


\noindent Paul Bleisch\index{Bleisch, Paul} (\mail{pbleisch@acm.org})\\
  Redid the debug system so that it would be much more
  flexible, so it could be easily disabled for production system, and
  so that messages for certain subsystems could be selectively
  enabled.

  Also contributed a first stab at a config file/command line parsing
  system.
 \medskip


\noindent Jim Brennan\index{Brennan, Jim} (\mail{jjb@foothill.net})\\
  Provided a big chunk of online space to store USA scenery for Flight Gear.
 \medskip

\noindent Bernie Bright\index{Bright, Bernie} (\mail{bbright@c031.aone.net.au})\\
  Many C++ style, usage, and implementation improvements, STL
  portability and much, much more.
 \medskip

\noindent Bernhard H. Buckel\index{Buckel, Bernhard H.}
(\mail{buckel@wmad95.mathematik.uni-wuerzburg.de})\\
  Contributed the README.Linux.  Coauthor of Installation
  and Getting Started (together with Michael Basler).
 \medskip

\noindent Gene Buckle\index{Buckle, Gene} (\mail{geneb@nwlink.com})\\
  A lot of work getting \FlightGear to compile with the \Index{MSVC}++
  compiler. Numerous hints on detailed improvements.
 \medskip

\noindent Oliver Delise \index{Delise, Oliver} (\mail{delise@rp-plus.de})\\
 FAQ Maintainer.
\medskip

\noindent Didier Chauveau\index{Chauveau, Didier} (\mail{chauveau@math.univ-mlv.fr})\\
  Provided some initial code to parse the 30 arcsec DEM files found at:

  \web{http://edcwww.cr.usgs.gov/landdaac/gtopo30/gtopo30.html}.
 \medskip


\noindent Jean-Francois Doue\index{Doue, Jean-Francois}\\
  Vector 2D, 3D, 4D and Matrix 3D and 4D inlined C++ classes.  (Based on
  Graphics Gems IV ed. Paul S. Heckbert)

\href{http://www.animats.com/simpleppp/ftp/public_html/topics/developers.html}{http://www.animats.com/simpleppp/ftp/public\_html/topics/developers.html}.
 \medskip

\noindent Francine Evans\index{Evans, Francine} (\mail{evans@cs.sunysb.edu})

\href{http://www.cs.sunysb.edu/~evans/stripe.html}{http://www.cs.sunysb.edu/\~{}evans/stripe.html}

  \noindent
  Wrote the GPL'd tri-striper.
 \medskip

\noindent Oscar Everitt\index{Everitt, Oscar} (\mail{bigoc@premier.net})\\
  Created single engine piston engine sounds as part of an F4U package
  for \Index{FS98}.  They are pretty cool and Oscar was happy to contribute
  them to our little project.
 \medskip

\noindent Jean-loup Gailly\index{Gailly, Jean-loup} and Mark Adler\index{Adler, Mark}
(\mail{zlib@quest.jpl.nasa.gov})\\
  Authors of the \Index{zlib library}.  Used for on-the-fly compression and
  decompression routines,

  \web{http://www.cdrom.com/pub/infozip/zlib/}.
 \medskip

\noindent Thomas Gellekum\index{Gellekum, Thomas} (\mail{tg@ihf.rwth-aachen.de})\\
  Changes and updates for compiling on \Index{FreeBSD}.
 \medskip

\noindent Jeff Goeke-Smith\index{Goeke-Smith, Jeff} (\mail{jgoeke@voyager.net})\\
  Contributed our first \Index{autopilot} (Heading Hold).
  Better autoconf check for external timezone/daylight variables.
 \medskip

\noindent Michael I. Gold\index{Gold, Michael, I.} (\mail{gold@puck.asd.sgi.com})\\
 Patiently answered questions on \Index{OpenGL}.
 \medskip

\noindent Charlie Hotchkiss\index{Hotchkiss, Charlie}
(\mail{chotchkiss@namg.us.anritsu.com})\\ Worked on improving and enhancing the
\Index{HUD} code.  Lots of code style tips and code tweaks\ldots
 \medskip

\noindent Bruce Jackson\index{Jackson, Bruce} (NASA) (\mail{e.b.jackson@larc.nasa.gov})

  \web{http://agcbwww.larc.nasa.gov/People/ebj.html}

 \noindent
   Developed the \Index{LaRCsim} code under funding by NASA which we use to provide the
   flight model. Bruce has patiently answered many, many questions.
 \medskip

\noindent Tom Knienieder\index{Knienieder, Tom} (\mail{knienieder@ms.netwing.at})\\
  Ported Steve Bakers's audio library\index{audio library} to Win32.
 \medskip

\noindent Reto Koradi\index{Koradi, Reto} (\mail{kor@mol.biol.ethz.ch})

\href{\web{http://www.mol.biol.ethz.ch/~kor}}{\web{http://www.mol.biol.ethz.ch/\~{}kor}}

\noindent
  Helped with setting up \Index{fog effects}.
 \medskip

\noindent Bob Kuehne\index{Kuehne, Bob} (\mail{rpk@sgi.com})\\
  Redid the Makefile system so it is simpler and more robust.
 \medskip

\noindent Vasily Lewis\index{Lewis, Vasily} (\mail{vlewis@woodsoup.org})

 \web{http://www.woodsoup.org}

 \noindent
  Provided computing resources and services so that the Flight Gear
  project could have real home.  This includes web services, ftp
  services, shell accounts, email lists, dns services, etc.
 \medskip


\noindent Christian Mayer\index{Mayer, Christian} (\mail{Vader@t-online.de})\\
 Working on multi-lingual conversion tools for fgfs.\\
 Contributed code to read msfs scenery textures.
 \medskip

\noindent Eric Mitchell\index{Mitchell, Eric} (\mail{mitchell@mars.ark.com})\\
  Contributed some topnotch scenery \Index{textures}.
 \medskip

\noindent Anders Morken\index{Morken, Anders} (\mail{amrken@online.no})\\
  Maintains the European mirror of the \FlightGear web pages.
 \medskip

\noindent Alan Murta\index{Murta, Alan} (\mail{amurta@cs.man.ac.uk})

  \web{http://www.cs.man.ac.uk/aig/staff/alan/software/}

  \noindent
  Created the Generic Polygon Clipping library.
 \medskip

\noindent Curt Olson\index{Olson, Curt} (\mail{curt@flightgear.org})\\
 Primary organization of the project. First implementation
 and modifications based on \Index{LaRCsim}. Besides putting together all
 the pieces provided by others mainly concentrating on the \Index{scenery
 engine} as well as the graphics stuff.
 \medskip

\noindent Robin Peel\index{Peel, Robin} (\mail{robinp@mindspring.com})\\
  Maintains worldwide airport and runway database for \FlightGear as we as X-Plane.
 \medskip


\noindent Friedemann Reinhard\index{Reinhard, Friedemann}
(\mail{mpt218@faupt212.physik.uni-erlangen.de})\\
  Development of textured instrument \Index{panel}.
 \medskip

\noindent Petter Reinholdtsen\index{Reinholdtsen, Petter} (\mail{pere@games.no})\\
  Incorporated the Gnu automake/autoconf system (with libtool).
  This should streamline and standardize the build process for all
  UNIX-like platforms.  It should have little effect on IDE type
  environments since they don't use the UNIX make system.
 \medskip

\noindent William Riley\index{Riley, William} (\mail{riley@technologist.com})\\
  Contributed code to add ''brakes''.
 \medskip

\noindent Paul Schlyter\index{Schlyter, Paul} (\mail{pausch@saaf.se})\\
  Provided Durk Talsma with all the information he needed to write the astro code.
 \medskip

\noindent Chris Schoeneman\index{Schoenemann, Chris} (\mail{crs@millpond.engr.sgi.com})\\
  Contributed ideas on audio support.
 \medskip

\noindent Jonathan R Shewchuk\index{Shewchuk, Jonathan}
(\mail{Jonathan\_R\_Shewchuk@ux4.sp.cs.cmu.edu})\\
  Author of the Triangle\index{triangle program} program.  Triangle
  is used to calculate the  Delauney triangulation of our irregular terrain.
 \medskip

\noindent Gordan Sikic\index{Sikic, Gordan} (\mail{gsikic@public.srce.hr})\\
  Contributed a \Index{Cherokee flight model} for \Index{LaRCsim}.  Currently is not
  working and needs to be debugged.  Use configure
  \texttt{-$\!$-with-flight-model=cherokee}
  to build the cherokee instead of the \Index{Navion}.
 \medskip

\noindent Michael Smith\index{Smith, Michael} (\mail{msmith99@flash.net})\\
  Contributed cockpit graphics, 3d models, logos, and other images.
  Project Bonanza

  \web{http://members.xoom.com/ConceptSim/index.html}.
 \medskip

\noindent
 \Index{U.\,S. Geological Survey}

\web{http://edcwww.cr.usgs.gov/doc/edchome/ndcdb/ndcdb.html}

 \noindent
  Provided geographic data used by this project.
 \medskip

\noindent Durk Talsma\index{Talsma, Durk} (\mail{pn\_talsma@macmail.psy.uva.nl})\\
  Accurate Sun, Moon, and Planets.  Sun changes color based on
  position in sky. Moon has correct phase and blends well into the
  sky.  Planets are correctly positioned and have proper magnitude. help with time
  functions, GUI, and other things.
 \medskip

\noindent Gary R. Van Sickle\index{van Sickle, Gary R.}
(\mail{tiberius@braemarinc.com})\\
  Contributed some initial \Index{GameGLUT} support and other fixes.
 \medskip

\noindent Norman Vine\index{Vine, Norman} (\mail{nhv@laserplot.com})\\
  Many performance optimizations throughout the code.  Many contributions
  and much advice for the scenery generation section.  Lots of Windows
  related contributions. Improved \Index{HUD}.
\medskip

\noindent Roland Voegtli\index{Voegtli, Roland} (\mail{webmaster@sanw.unibe.ch})\\
 Contributed great photorealistic textures.
\medskip


\noindent Carmelo Volpe\index{Volpe, Carmelo} (\mail{carmelo.volpe@csb.ki.se})\\
  Porting \FlightGear to the \Index{Metro Works} development environment
  (PC/Mac).
 \medskip

\noindent Darrell Walisser\index{Walisser, Darrell} (\mail{dwaliss1@purdue.edu})\\
 Contributed a large number of changes to porting \FlightGear to the Metro Works
 development environment (PC/Mac). Finally produced the first MacIntosh port.
\medskip


\noindent Robert Allan Zeh\index{Zeh, Allan} (\mail{raz@cmg.FCNBD.COM})\\
  Helped tremendously in figuring out the \Index{Cygnus} Win32 compiler and
  how to link with .dll's.  Without him the first run-able Win32
  version of \FlightGear would have been impossible.

\section{What remains to be done}
At first: If you read (and, maybe, followed) this guide until this
point you may probably agree that \FlightGear\hspace{-2mm}, even
in its present state, is not at all for the birds. It is already a
flight simulator which has a flight model, a plane, terrain
scenery, texturing and simple controls.

Despite, \FlightGear needs -- and gets -- further development. Except internal tweakings,
there are several fields where \FlightGear needs basics improvement and development.

A first direction is adding \Index{airports}, streets, and more things bringing Scenery
to real life.

Second, the \Index{panel} needs further improvement including more working gauges.

Besides, there should be support for adding more \Index{planes} and for implementing
corresponding flight models differing from the \Index{Navion}.

Another task is further implementation of the \Index{menu system}, which should not be
too hard with the basics being working now.

A main stream of active development concerns weather. At present there is simply none: no
clouds, no rain, no wind. But there sure will be.

There are already people working in all of these directions. If you're a programmer and
think you can contribute, you are invited to do so.

\subsection*{Achnowledgements}
Obviously this document could not have been written without all
those contributors mentioned above making \FlightGear a reality.

Beyond this we would like to say special thanks to Curt
Olson,\index{Olson, Curt} whose numerous scattered Readmes,
Thanks, Webpages, and personal eMails were of special help to us
and were freely exploited in the making of this booklet.

Next, we gained a lot of help and support from Steve Baker \index{Baker, Steve} and
Norman Vine\index{Vine, Norman}. Moreover, we would like to thank Steve
Baker\index{Baker, Steve} for a careful reading and for numerous hints on the first draft
of this guide.

Further, we would like to thank Kai Troester\index{Troester, Kai} for donating the
solution of some of his compile problems to Chapter \ref{missed}.

%% Revision 0.00  1998/09/08  michael
%% Initial revision for version 0.53.
%% Revision 0.01  1998/09/20  michael
%% several extensions and corrections
%% revision 0.10  1998/10/01  michael
%% final proofreading for release
%% revision 0.11  1998/11/01  michael
%% corrections on audio library, getting started
%% revision 0.12  1999/03/07  michael
%% Updated Credits
%% revision 0.20  1999/06/04  michael
%% added O. Delise, Ch. Mayer, R. Peel, R. Voegtli, several updates
