%
% `Stars.tex' -- describes our procedure for drawing stars
%
%  Written by Curtis Olson.  Started December, 1997.
%
% $Id$
%------------------------------------------------------------------------


\documentclass[12pt]{article}

\usepackage{anysize}
\papersize{11in}{8.5in}
\marginsize{1in}{1in}{1in}{1in}

\usepackage{amsmath}

\usepackage{epsfig}

\usepackage{setspace}
\onehalfspacing

\usepackage{url}


\begin{document}


\title{
  Flight Gear Stars Representation and Rendering.
}


\author{
    Curtis L. Olson\\ 
    (\texttt{curt@me.umn.edu})
}


\maketitle


\section{Introduction}

Flight Gear attempts to render the top several hundred stars in the
correct position in the sky for the current time and position as will
as rendering these stars with the correct ``magnitude''.

This document will give a quick overview of our approach.

\section{Resources}

\subsubsection{XEphem}

The set of bright stars was extracted from the xephem program.  For a
full list of features and the latest news, please see the xephem home
page at \url{http://iraf.noao.edu/~ecdowney/xephem.html}.  The XEphem
star list was itself taken from the Yale Bright Star Catalog.

\begin{quote}
  Based on the 5th Revised edition of the Yale Bright Star Catalog,
  1991, from ftp://adc.gsfc.nasa.gov/pub/adc/archives/catalogs/5/5050.
  
  Only those entries with a Bayer and/or Flamsteed number are retained
  here.
  
  Format: Constellation BayerN-Flamsteed, as available. Bayer is
  truncated as required to enforce a maximum total length of 13
  imposed within xephem.
  
  Common names were then overlayed by closest position match from
  hand-edited a list supplied by Robert Tidd (inp@violet.berkeley.edu)
  and Alan Paeth (awpaeth@watcgl.waterloo.edu)
\end{quote}

The author of XEphem, Elwood Downey (ecdowney@noao.edu), was very
instrumental in helping me understand sidereal time, accurate star
placement, and even contributed some really hairy sections of code.
Thanks Elwood!


\section{Terminology and Definitions}

The following information is repeated verbatim from
\url{http://www.lclark.edu/~wstone/skytour/celest.html}: If you are
interested in these sorts of things I urge you to visit this site.  It
contains much more complete information.

\subsection{Celestial Measurements} 

Although we know that the objects we see in the sky are of different
sizes and at different distances from us, it is convenient to
visualize all the objects as being attached to an imaginary sphere
surrounding the Earth.  From our vantage point, the sky certainly
looks like a dome (the half of the celestial sphere above our local
horizon).  The celestial sphere is mapped in Right Ascension (RA) and
Declination (Dec).

\subsubsection{Declination}

Declination is the celestial equivalent of latitude, and is simply the
Earth's latitude lines projected onto the celestial sphere. A star
that can be directly overhead as seen from the Earth's Equator (0
degrees latitude) is said to be on the Celestial Equator, and has a
declination of 0 degrees . The North Star, Polaris, is very nearly
overhead as seen from the North Pole (90 degrees North latitude). The
point directly over the North Pole on the celestial sphere is called
the North Celestial Pole, and has a declination of +90 degrees .
Northern declinations are given positive signs, and southern
declinations are given negative signs. So, the South Celestial Pole
has a declination of -90 degrees .

\subsubsection{Right Ascension \& Sidereal Time}

Right Ascension is the equivalent of longitude, but since the Earth
rotates with respect to the celestial sphere we cannot simply use the
Greenwich Meridian as 0 degrees RA. Instead, we set the zero point as
the place on the celestial sphere where the Sun crosses the Celestial
Equator (0 degrees Dec) at the vernal (spring) equinox.  The arc of
the celestial sphere from the North Celestial Pole through this point
to the South Celestial Pole is designated as Zero hours RA. Right
Ascension increases eastward, and the sky is divided up into 24
hours. This designation is convenient because it represents the
sidereal day, the time it takes for the Earth to make one rotation
relative to the celestial sphere. If you pointed a telescope (with no
motor drive) at the coordinates (RA=0h, Dec=0 degrees ), and came back
one hour later, the telescope would then be pointing at (RA=1h, Dec=0
degrees ). Because the Earth's revolution around the Sun also
contributes to the apparent motion of the stars, the day we keep time
by (the solar day) is about four minutes longer than the sidereal
day. So, if you pointed a telescope at (RA=0h, Dec=0 degrees ) and
came back 24 hours later, the telescope would now be pointing at
(RA=0h 4m, Dec=0 degrees). A consequence is that the fixed stars
appear to rise about four minutes earlier each day.


\subsection{Implementation}

Here is a brief overview of how stars were implemented in Flight Gear.
The right ascension and declination of each star is used to build a
structure of point objects to represent the stars.  The magnitude is
mapped into a color on the gray/white continuum.  The points are
positioned just inside the far clip plane.  When rendering the stars,
this structure (display list) is rotated about the $Z$ axis by the
current sidereal time and translated to the current view point.

\subsubsection{Data file format}

The star information is stored in a simple data file called
``Stars.dat'' with the following comma delimited format: name, right
ascension(radians), declination(radians), magnitude(smaller is
brighter).  Here is an extract of the data file:

\begin{verbatim}
Sirius,1.767793,-0.266754,-1.460000
Canopus,1.675305,-0.895427,-0.720000
Arcturus,3.733528,0.334798,-0.040000
Rigil Kentaurus,3.837972,-1.032619,-0.010000
Vega,4.873563,0.676902,0.030000
Capella,1.381821,0.802818,0.080000
Rigel,1.372432,-0.136107,0.120000
Procyon,2.004082,0.091193,0.380000
Achernar,0.426362,-0.990707,0.460000
\end{verbatim}

\subsubsection{Building the display list}

The display list is built up from a collection of point objects as the
star data file is loaded.  For each star, the magnitude is mapped into
a brightness value from 0.0 to 1.0 with 1.0 being the brightest.  Our
coordinate system is described in the coordinate system document: Z
points towards the North pole, X out of the 0 degree meridian at the
equator, and Y out at the Indian ocean at the equator.  Given this
coordinate system, the position of each star at 0:00H sidereal time is
calculated as follows:

\begin{align}
x &= \mathrm{distance} * \cos(\mathrm{rightascension}) * 
     \cos(\mathrm{declination}) \\
y &= \mathrm{distance} * \sin(\mathrm{rightascension}) * 
     \cos(\mathrm{declination}) \\
z &= \mathrm{distance} * \sin(\mathrm{declination})
\end{align}

\subsubsection{Transformations and Rendering}

The tricky part about rendering the stars is calculating sidereal time
correctly.  Here's where Elwood Downey saved my butt.  0:00H sidereal
time aligns with 12:00 noon GMT time on March 21 of every year.  After
that they diverge by about 4 minutes per day.  The solar day is
approximately 4 minutes longer than the side real day.  Once you know
the proper sidereal time, you simply translate the center of the star
structure to the current view point, then rotate this structure about
the Z axis by the current sidereal time.

The stars are drawn out by the far clip plane so they don't occult
anything.  They are translated using the same x/y/z as the eye point
so that you can never fly any closer to them.

\end{document}


%------------------------------------------------------------------------
% $Log$
% Revision 1.1  1999/04/05 21:32:34  curt
% Initial revision
%
% Revision 1.1  1999/02/14 19:10:47  curt
% Initial revisions.
%
