%%
%% getstart.tex -- Flight Gear documentation: Installation and Getting Started
%% Chapter file
%%
%% Written by Michael Basler, started September 1998.
%%
%% Copyright (C) 1999 Michael Basler (pmb@knUUt.de)
%%
%%
%% This program is free software; you can redistribute it and/or
%% modify it under the terms of the GNU General Public License as
%% published by the Free Software Foundation; either version 2 of the
%% License, or (at your option) any later version.
%%
%% This program is distributed in the hope that it will be useful, but
%% WITHOUT ANY WARRANTY; without even the implied warranty of
%% MERCHANTABILITY or FITNESS FOR A PARTICULAR PURPOSE.  See the GNU
%% General Public License for more details.
%%
%% You should have received a copy of the GNU General Public License
%% along with this program; if not, write to the Free Software
%% Foundation, Inc., 675 Mass Ave, Cambridge, MA 02139, USA.
%%
%% $Id: getstart.tex,v 0.12 1999/03/07 michael
%% (Log is kept at end of this file)

%%%%%%%%%%%%%%%%%%%%%%%%%%%%%%%%%%%%%%%%%%%%%%%%%%%%%%%%%%%%%%%%%%%%%%%%%%%%%%%%%%%%%%%%%%%%%%%%%
\chapter{Preflight: Installing \FlightGear \label{prefligh}}
%%%%%%%%%%%%%%%%%%%%%%%%%%%%%%%%%%%%%%%%%%%%%%%%%%%%%%%%%%%%%%%%%%%%%%%%%%%%%%%%%%%%%%%%%%%%%%%%%
\markboth{\thechapter.\hspace*{1mm}
PREFLIGHT}{\thesection\hspace*{1mm} INSTALLING THE BINARIES}

\section{Installing the Binaries}\index{binaries!installation}
You can skip this section and go to the installation of scenery in
case you built \FlightGear along the lines describes during the
previous chapter. If you did not and you're jumping in here your
first step consists in installing the binaries. At present, there
are only pre-compiled \Index{binaries} available for
\Index{Windows 98/NT} while in principle it might be possible to
create (statically linked) binaries for \Index{Linux} as well.

The following supposes you are on a Windows 98 or Windows
NT\index{Windows 98/NT} system. Installing the binaries is quite
simple. Go to the \FlightGear downloads page

 \web{http://www.flightgear.org/Downloads/}

 \noindent
and get the latest binaries from the binaries subdirectory named

\texttt{fg-win32-bin-X.XX.exe}

 \noindent
and unpack them via double clicking. This will create a directory \texttt{FlightGear}
with several subdirectories. You are done.

\section{Installing \Index{Support files}}

Independent on your operating system and independent on if you
built the binaries yourself or installed the precompiled ones as
described above you will need \Index{scenery}, \Index{texture},
and \Index{sound} files. A basic package of all these is contained
in the binaries directory mentioned above as

 \texttt{fgfs-base-X.XX}.

 \noindent
 Preferably, you may want to download the \texttt{.tar.gz} version
if you are working under \Index{Linux}/\Index{UNIX} and the \texttt{.exe} version if you
are under \Index{Windows 98/NT}. Make sure you get the \textbf{most recent} version.

If you're working under \Index{Linux} or \Index{UNIX}, unpack the
previously downloaded file with

        \texttt{tar xvfz fgfs-base-X.XX.tar.gz},

 \noindent
while under \Index{Windows 98/NT} just double click on the file (being situated in the
root of your \FlightGear drive.).

This already completes installing \FlightGear and should enable
you to invoke the program.

Some more scenery which, however, is not a substitute for the
package mentioned above but rather is based on it can be found in
the scenery subdirectory under

 \web{http://www.flightgear.org/Downloads/}

 \noindent
These may be older versions which may or may not work with the
most recent binaries.

In addition, there is a complete set of \Index{USA Scenery files}
available created by Curt Olson\index{Olson, Curt} which can be
downloaded from

\web{ftp://ftp.kingmont.com/pub/kingmont/index.html}.

 \noindent
The complete set covers several 100's of MBytes. Thus, Curt
provides the complete set on CD-ROM for those who really would
like to fly over all of the USA. For more detail, check the
remarks in the downloads page above.

Finally, the binaries directory mentioned contain the complete
\FlightGear documentation including a .pdf version of this
\textit{Installation and Getting Started} guide intended for
pretty printing using Adobe's Acrobat reader being available from



\web{http://www.adobe.com/acrobat}.

\noindent
 on any printer.

%% Revision 0.00  1998/09/08  michael
%% Initial revision for version 0.53.
%% Revision 0.01  1998/09/20  michael
%% several extensions and corrections
%% revision 0.10  1998/10/01  michael
%% final proofreading for release
%% revision 0.11  1998/11/01  michael
%% support files section completely re-written
